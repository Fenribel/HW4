\documentclass[12pt,a4paper]{scrartcl}
\usepackage[utf8]{inputenc}
\usepackage[english,russian]{babel}
\usepackage{indentfirst}
\usepackage{misccorr}
\usepackage{graphicx}
\usepackage{amsmath}
\usepackage{bm}
\usepackage{listings}
\usepackage{graphicx}
\usepackage{xcolor}
\usepackage{hyperref}
\usepackage[normalem]{ulem}
\DeclareGraphicsExtensions{.pdf,.png,.jpg}
\author{А. А. Галиуллин}
\lstset{
	language=Python,
	basicstyle=\ttfamily,
	columns=fullflexible,
	frame=single,
	breaklines=true,
}
\begin{document}
	
	\begin{center}
		\large
		Домашняя работа \\
		Выполнил Галиуллин Арслан, 1 курс факультета физики, группа 171. \\
		1233550v@mail.ru \\
		Желаемая оценка - 10.
	\end{center}

	Домашняя работа по дифференциированию и интегрированию.\\Все коды программы лежат на \href{https://github.com/Fenribel/HW4.git}{гитхабе}.
	\section{Найдём производную синуса }
		Посчитаем её по формуле $f'(x) = \frac{f(x+h) - f(x-h)}{2h}$. Погрешность будет составлять $O(h^2)$.

		\begin{center}
			\includegraphics[scale=0.8]{figure_1} \\
		\end{center}
		Видно, что производная и косинус визуально совпали. Погрешность наименьшая при $h = 10^{-5.1}$, её квадрат равен $\Delta^2 = 10^{-11}$.
	\section{Найдём производную экспоненты }
		\begin{center}
			\includegraphics[scale=0.73]{figure_2} \\
		\end{center}
	$h = 10^{-6.9}$, $\Delta^2 = 10^{-5}$
	\section{Зависимость погрешности от h }
		\textbf{sin(x)}
			\begin{center}
				\includegraphics[scale=0.73]{figure_3} \\
			\end{center}
		\textbf{exp(x)}
			\begin{center}
				\includegraphics[scale=0.74]{figure_4} \\
			\end{center}
	\section{Посчитаем те же производные точнее}
		$f'(x) = \frac{2 f(x+h) + f(x+2h) - f(x-2h) - 2 f(x-h)}{4h}$, точность $O(h^3)$.
		\begin{center}
			\includegraphics[scale=0.8]{figure_5} \\
		\end{center}
		$h = 10^{-5.1}$, $\Delta^2 = 10^{-11}$ - то же самое.
				\begin{center}
			\includegraphics[scale=0.74]{figure_6} \\
		\end{center}
		$h = 10^{-6.9}$, $\Delta^2 = 10^{-5}$ - то же самое.
	\section{Вторые производные}
		$f''(x) = \frac{f(x+h) - 2f(x) + f(x-h)}{h^2}$, погрешность $O(h^2)$.
		\begin{center}
			\includegraphics[scale=0.74]{figure_7} \\
		\end{center}
		$h = 10^{-3.7}$, $\Delta^2 = 10^{-9}$ - погрешность растёт.
		\begin{center}
			\includegraphics[scale=0.73]{figure_8} \\
		\end{center}
		$h = 10^{-3.7}$, $\Delta^2 = 10^{-3}$ - погрешность растёт.
	\section{Интегрирование}
	Будем считать интеграл $I = \int\limits_a^b f(x)dx =  \sum_{i=1}^N  f({x_i}) h$, где $h = \frac{b-a}{N}$.
	\begin{center}
		\includegraphics[scale=0.73]{figure_9} \\
	\end{center}
	$\Delta^2 = 0.5$ - погрешность ощутимая из-за того, что функция быстро растёт. Она считалась после сдвига графика интеграла вверх на единицу. Относительная погрешность $\varepsilon^2 = 10^{-3}$
	\section{Попробуем теперь продифференцировать и проинтегрировать функцию}
		\begin{center}
			\includegraphics[scale=0.68]{figure_10} \\
		\end{center}
		$\Delta^2 = 0.6$, $\varepsilon^2 = 10^{-3}$
		\begin{center}
			\includegraphics[scale=0.68]{figure_11} \\
		\end{center}
		$\Delta^2 = 10^{-3}$
	\section{Посчитаем какой-нибудь интеграл, а потом сравним с тем, что считает Mathematica}
	 $I = \int\limits_{0.524 (\pi/6)}^{1.047 (\pi/3)} \frac{sin^2(x)}{cos^2(x) \sqrt{tan(x)}}dx =
	 \begin{cases}
	    0.581895835761243,  & \text{программа} \\
	 	0.5818958362420757, & Mathematica
	 	
	 \end{cases}$
	 \\
	 Относительная разница $\varepsilon^2 = 10^{-18}$. Весьма точно.
	 \\
	 Разбиение при подсчёте было на $10^9$ точек. Считало долго.
	\section{Интеграл 1/x}
	Чтобы посчитать интеграл функции, уходящей симметрично на бесконечность (например, $1/x$), нужно считать интеграл так, чтобы обе бесконечности занулились.
	Для 1/x он будет выглядеть так: 
	$I = \int\limits_{-1}^1 \frac{1}{x}dx =  \sum_{i=1}^N  \frac{f(x+h/2)+f(x-h/2)}{2} h = 10^{-4}$
\end{document}